\documentclass[12pt]{article}
\usepackage[margin=1in]{geometry}
\usepackage{graphicx}
\usepackage{amsmath}
\usepackage{algorithm}
\usepackage{algorithmic}
\usepackage{algpseudocode}
\usepackage{listings}
\usepackage{hyperref}
\usepackage{xcolor}
\usepackage{setspace}
\usepackage{float}
\usepackage{caption}
\usepackage{subcaption}
\usepackage{appendix}

% Define colors for syntax highlighting
\definecolor{codegreen}{rgb}{0,0.6,0}
\definecolor{codegray}{rgb}{0.5,0.5,0.5}
\definecolor{codepurple}{rgb}{0.58,0,0.82}
\definecolor{backcolour}{rgb}{0.95,0.95,0.92}

% Python code style
\lstdefinestyle{mystyle}{
    backgroundcolor=\color{backcolour},   
    commentstyle=\color{codegreen},
    keywordstyle=\color{magenta},
    numberstyle=\tiny\color{codegray},
    stringstyle=\color{codepurple},
    basicstyle=\ttfamily\footnotesize,
    breakatwhitespace=false,         
    breaklines=true,                 
    captionpos=b,                    
    keepspaces=true,                 
    numbers=left,                    
    numbersep=5pt,                  
    showspaces=false,                
    showstringspaces=false,
    showtabs=false,                  
    tabsize=2
}
\lstset{style=mystyle}

\title{Comparative Analysis of LCS-FIG Algorithms:\\FIG-DP, RMQ-FIG, and Greedy Approaches}
\author{Mohammad Sadegh Sirjani | ZXL708}
\date{\today}

\begin{document}

\maketitle
% \tableofcontents
\newpage

\begin{abstract}
This report presents a comprehensive comparison of three algorithms for solving the Longest Common Subsequence with Fixed-length Indel Gaps (LCS-FIG) problem: the dynamic programming approach (FIG-DP), the Range Minimum Query optimization (RMQ-FIG), and a novel greedy approach. We analyze their performance characteristics, solution quality, and practical applicability across various input sizes and gap constraints.
\end{abstract}

\section{Introduction}
The LCS-FIG problem extends the classical LCS problem by allowing fixed-length indel gaps. This extension has practical applications in computational biology, particularly in DNA sequence alignment. Our analysis compares three distinct approaches:
\begin{itemize}
    \item \textbf{FIG-DP}: A dynamic programming solution guaranteeing optimal results
    \item \textbf{RMQ-FIG}: An optimization using Range Minimum Queries
    \item \textbf{Greedy LCS-FIG}: A heuristic approach prioritizing speed over optimality
\end{itemize}

\section{Performance Analysis}
\subsection{Experimental Setup}
Our experimental evaluation used:
\begin{itemize}
    \item Sequence lengths: [100, 500, 1000, 2000, 5000]
    \item Gap constraints (K): [2, 5, 10, 20, 40]
    \item Random DNA sequences (A, C, G, T)
    \item Multiple trials per configuration
\end{itemize}

\subsection{Time Complexity Analysis}
\begin{figure}[H]
    \centering
    \includegraphics[width=0.9\textwidth]{./images/final-report/performance_comparison.png}
    \caption{Performance Comparison of All LCS-FIG Algorithms}
    \label{fig:performance_comparison}
\end{figure}

As shown in Figure \ref{fig:performance_comparison}, we observe significant performance differences:
\begin{itemize}
    \item \textbf{FIG-DP}: Shows O(n²K) time complexity, with execution time ranging from 0.011s (n=100, K=2) to 393.57s (n=5000, K=40)
    \item \textbf{RMQ-FIG}: Demonstrates O(n²) complexity, with times from 0.018s (n=100, K=2) to 454.95s (n=5000, K=10)
    \item \textbf{Greedy}: Exhibits near-linear O(nK) growth, with execution times from 7.3×10⁻⁶s (n=100, K=40) to 1.65×10⁻³s (n=5000, K=2)
\end{itemize}

The greedy algorithm outperforms the other approaches by several orders of magnitude, especially as sequence length and gap size increase.

\subsection{Relative Performance Analysis}

\begin{figure}[H]
    \centering
    \includegraphics[width=0.8\textwidth]{./images/final-report/relative_performance_figdp_vs_greedy.png}
    \caption{Relative Performance: FIG-DP vs Greedy}
    \label{fig:figdp_vs_greedy}
\end{figure}

Figure \ref{fig:figdp_vs_greedy} demonstrates the dramatic performance difference between FIG-DP and Greedy approaches:
\begin{itemize}
    \item For K=2, FIG-DP is 431-15,021× slower than Greedy
    \item For K=40, this ratio increases to 12,813-830,210×
    \item Performance gap widens with both sequence length and gap size
\end{itemize}

\begin{figure}[H]
    \centering
    \includegraphics[width=0.8\textwidth]{./images/final-report/relative_performance_figdp_vs_rmq.png}
    \caption{Relative Performance: FIG-DP vs RMQ-FIG}
    \label{fig:figdp_vs_rmq}
\end{figure}

The performance comparison between FIG-DP and RMQ-FIG (Figure \ref{fig:figdp_vs_rmq}) shows:
\begin{itemize}
    \item For K≤10, RMQ-FIG is typically slower than FIG-DP (ratio < 1.0)
    \item For K≥20, FIG-DP is consistently slower than RMQ-FIG (ratio > 1.0)
    \item For K=40, FIG-DP is 10-16\% slower than RMQ-FIG across all input sizes
\end{itemize}

\begin{figure}[H]
    \centering
    \includegraphics[width=0.8\textwidth]{./images/final-report/relative_performance_rmq_vs_greedy.png}
    \caption{Relative Performance: RMQ-FIG vs Greedy}
    \label{fig:rmq_vs_greedy}
\end{figure}

The RMQ-FIG vs Greedy comparison (Figure \ref{fig:rmq_vs_greedy}) reveals:
\begin{itemize}
    \item For K=2, RMQ-FIG is 693-23,663× slower than Greedy
    \item For K=40, this difference increases to 11,602-714,908×
    \item For large inputs with K=10, RMQ-FIG can be 375,158× slower than Greedy
\end{itemize}

\subsection{Space Complexity}
Memory usage varies significantly between algorithms:
\begin{itemize}
    \item \textbf{FIG-DP}: Requires 52.9-733.0 MB for sequences up to length 5000
    \item \textbf{RMQ-FIG}: Uses 53.0-733.7 MB, slightly higher than FIG-DP due to additional data structures
    \item \textbf{Greedy}: Negligible memory usage, measured at less than 1 MB across all test cases
\end{itemize}

\subsection{Solution Quality}
\begin{figure}[H]
    \centering
    \includegraphics[width=0.8\textwidth]{./images/final-report/solution_quality.png}
    \caption{Greedy Algorithm Solution Quality vs Optimal Solution}
    \label{fig:solution_quality}
\end{figure}

Figure \ref{fig:solution_quality} illustrates the quality ratio (Greedy/Optimal) across different K values:
\begin{itemize}
    \item For K=2: 18.3-19.7\% of optimal solution length (avg 18.8\%)
    \item For K=5: 15.8-17.6\% of optimal solution length (avg 16.5\%)
    \item For K=10: 11.4-12.6\% of optimal solution length (avg 12.1\%)
    \item For K=20: 8.5-11.5\% of optimal solution length (avg 9.5\%)
    \item For K=40: 5.4-6.9\% of optimal solution length (avg 6.0\%)
\end{itemize}

This demonstrates a clear inverse relationship between gap size and solution quality in the greedy approach.

\section{Algorithm Recommendations}
Based on our comprehensive analysis, we recommend:

\subsection{Use FIG-DP when:}
\begin{itemize}
    \item Optimal solutions are required
    \item Input sequences are short (n ≤ 1000)
    \item Memory constraints are not critical
    \item K value is small (≤ 10)
    \item Applications: Reference implementations, benchmark creation
\end{itemize}

\subsection{Use RMQ-FIG when:}
\begin{itemize}
    \item Optimal solutions are required
    \item Medium-sized sequences (1000 < n ≤ 5000)
    \item Balanced time-space trade-off is needed
    \item K value is large (> 10)
    \item Applications: Production systems with moderate load
\end{itemize}

\subsection{Use Greedy LCS-FIG when:}
\begin{itemize}
    \item Speed is critical
    \item Long sequences (n > 5000)
    \item Approximate solutions are acceptable
    \item Memory is limited
    \item K is small (≤ 5), where solution quality remains above 15\%
    \item Applications: Real-time systems, large-scale processing
\end{itemize}

\section{Trade-offs and Considerations}

\subsection{Time-Quality Trade-off}
\begin{itemize}
    \item FIG-DP: 100\% accuracy, slowest performance (avg 29.5s for n=5000, K=20)
    \item RMQ-FIG: 100\% accuracy, moderate performance (avg 26.1s for n=5000, K=20)
    \item Greedy: 5-20\% accuracy depending on K, fastest performance (avg 0.0003s for n=5000, K=20)
\end{itemize}

\subsection{Memory-Performance Trade-off}
\begin{itemize}
    \item FIG-DP: High memory usage (730 MB for n=5000, K=20), predictable performance
    \item RMQ-FIG: High memory usage (731 MB for n=5000, K=20), better performance for large K
    \item Greedy: Negligible memory usage, fastest execution
\end{itemize}

\subsection{Gap Constraint Impact}
\begin{itemize}
    \item Small K (≤ 10): FIG-DP outperforms RMQ-FIG, Greedy quality ~12-19\%
    \item Medium K (11-20): RMQ-FIG begins to outperform FIG-DP, Greedy quality ~9-12\%
    \item Large K (> 20): RMQ-FIG significantly outperforms FIG-DP, Greedy quality < 9\%
\end{itemize}

\section{Conclusion}
The choice of algorithm depends primarily on the specific requirements of the application:
\begin{enumerate}
    \item For applications requiring guaranteed optimal solutions and handling smaller sequences, FIG-DP remains the best choice.
    \item For balanced performance and optimal solutions with medium-sized sequences, RMQ-FIG provides an excellent compromise.
    \item For large-scale applications where approximate solutions are acceptable, the Greedy approach offers superior performance and minimal memory usage.
\end{enumerate}

Future work should focus on:
\begin{itemize}
    \item Parallel implementations of all three algorithms
    \item Hybrid approaches combining greedy and dynamic programming
    \item Optimization for specific sequence characteristics
\end{itemize}

\newpage

\appendix
\section{FIG-DP Analysis Report}
\documentclass[12pt]{article}
\usepackage[margin=1in]{geometry}
\usepackage{graphicx}
\usepackage{amsmath}
\usepackage{algorithm}
\usepackage{algpseudocode}
\usepackage{listings}
\usepackage{hyperref}
\usepackage{xcolor}
\usepackage{setspace}
\usepackage{float}
\usepackage{caption}
\usepackage{subcaption}

% Define colors for syntax highlighting
\definecolor{codegreen}{rgb}{0,0.6,0}
\definecolor{codegray}{rgb}{0.5,0.5,0.5}
\definecolor{codepurple}{rgb}{0.58,0,0.82}
\definecolor{backcolour}{rgb}{0.95,0.95,0.92}

% Python code style
\lstdefinestyle{mystyle}{
    backgroundcolor=\color{backcolour},   
    commentstyle=\color{codegreen},
    keywordstyle=\color{magenta},
    numberstyle=\tiny\color{codegray},
    stringstyle=\color{codepurple},
    basicstyle=\ttfamily\footnotesize,
    breakatwhitespace=false,         
    breaklines=true,                 
    captionpos=b,                    
    keepspaces=true,                 
    numbers=left,                    
    numbersep=5pt,                  
    showspaces=false,                
    showstringspaces=false,
    showtabs=false,                  
    tabsize=2
}
\lstset{style=mystyle}

\title{Longest Common Subsequence with Gap Constraints (LCS-FIG)\\
Implementation and Performance Analysis Report}
\author{Mohammad Sadegh Sirjani}
\date{\today}

\begin{document}
\maketitle

\begin{abstract}
This report presents a detailed implementation and analysis of the Longest Common Subsequence with Gap Constraints (LCS-FIG) algorithm. The algorithm extends the classic LCS problem by introducing gap constraints between consecutive matches, making it particularly suitable for DNA sequence analysis. We present a dynamic programming solution with O(nm) time and space complexity, along with comprehensive experimental results and performance analysis. The implementation demonstrates efficient scaling with input size and practical applicability for bioinformatics applications.
\end{abstract}

\section{Introduction}

\subsection{Background}
The Longest Common Subsequence (LCS) problem is a fundamental problem in computer science with applications in bioinformatics, text comparison, and version control systems. The LCS-FIG variant adds gap constraints between consecutive matches, making it particularly relevant for biological sequence analysis where spacing between matching elements is significant.

\subsection{Problem Significance}
In DNA sequence analysis, identifying common subsequences with controlled gaps is crucial for:
\begin{itemize}
    \item \textbf{Gene identification and comparison}: Enables detection of similar genetic regions across different species, helping identify conserved functional elements and evolutionary relationships between organisms.
    \item \textbf{Regulatory sequence analysis}: Facilitates the study of DNA regions that control gene expression by identifying common patterns in promoter sequences and other regulatory elements.
    \item \textbf{Evolutionary relationship studies}: Helps reconstruct phylogenetic trees and understand species relationships by analyzing conserved sequence patterns across different organisms.
    \item \textbf{Mutation pattern detection}: Aids in identifying genetic variations and understanding mutation patterns by comparing sequences from different individuals or populations.
\end{itemize}

\section{Problem Definition}

\subsection{Formal Definition}
Given:
\begin{itemize}
    \item Two sequences X[1..n] and Y[1..m]
    \item A gap constraint parameter K
\end{itemize}

Find: The longest common subsequence Z such that:
\begin{equation}
    \forall i > 1: pos_X(Z[i]) - pos_X(Z[i-1]) \leq K + 1
\end{equation}
\begin{equation}
    \forall i > 1: pos_Y(Z[i]) - pos_Y(Z[i-1]) \leq K + 1
\end{equation}

where pos_X(z) and pos_Y(z) denote the positions of element z in sequences X and Y respectively.

\section{Algorithm Description}

\subsection{Dynamic Programming Formulation}

Let T[i,j] represent the length of the LCS-FIG ending at positions i and j in sequences X and Y respectively.

\begin{equation}
T[i,j] = \begin{cases}
    T[i-K-1,j-K-1] + 1 & \text{if } X[i]=Y[j] \text{ and } i,j > K+1\\
    1 & \text{if } X[i]=Y[j] \text{ and } (i \leq K+1 \text{ or } j \leq K+1)\\
    \max(T[i-1,j], T[i,j-1]) & \text{otherwise}
\end{cases}
\end{equation}

\subsection{Algorithm Implementation}

\begin{algorithm}
\caption{LCS-FIG Dynamic Programming Solution}
\begin{algorithmic}[1]
\Require Sequences X[1..n], Y[1..m], gap constraint K
\Ensure Length of LCS-FIG and the subsequence
\State Initialize T[0..n,0..m] with zeros
\State Initialize Parent[0..n,0..m] for backtracking
\For{i = 1 to n}
    \For{j = 1 to m}
        \If{X[i] = Y[j]}
            \If{i > K+1 and j > K+1}
                \State T[i,j] = T[i-K-1,j-K-1] + 1
                \State Parent[i,j] = DIAGONAL\_GAP
            \Else
                \State T[i,j] = 1
                \State Parent[i,j] = START
            \EndIf
        \Else
            \If{T[i-1,j] ≥ T[i,j-1]}
                \State T[i,j] = T[i-1,j]
                \State Parent[i,j] = UP
            \Else
                \State T[i,j] = T[i,j-1]
                \State Parent[i,j] = LEFT
            \EndIf
        \EndIf
    \EndFor
\EndFor
\State \Return ReconstructSolution(Parent, X, Y)
\end{algorithmic}
\end{algorithm}

\section{Implementation Details}

\subsection{Data Structures}
The implementation uses the following key data structures:

\begin{lstlisting}[language=Python, caption=Core Data Structures]
class FIGDP:
    def __init__(self, seq1: str, seq2: str, k: int):
        self.seq1 = seq1
        self.seq2 = seq2
        self.n = len(seq1)
        self.m = len(seq2)
        self.k = k
        self.dp_table = np.zeros((n+1, m+1), dtype=np.int32)
        self.parent = np.zeros((n+1, m+1), dtype=np.int32)
\end{lstlisting}

\subsection{Optimization Techniques}

\subsubsection{Memory Optimizations}
\begin{itemize}
    \item \textbf{Use of NumPy arrays for efficient memory layout}: Implements contiguous memory storage and vectorized operations, resulting in faster access times and better cache utilization compared to Python lists.
    \item \textbf{Int32 data type for reduced memory footprint}: Uses 32-bit integers instead of Python's default dynamic typing, reducing memory usage by up to 50\% for large sequences.
    \item \textbf{Contiguous memory allocation for better cache utilization}: Ensures that array elements are stored in consecutive memory locations, maximizing CPU cache efficiency and reducing memory access times.
\end{itemize}

\subsubsection{Computational Optimizations}
\begin{itemize}
    \item \textbf{Single-pass solution reconstruction}: Implements an efficient backtracking algorithm that reconstructs the solution in a single pass through the parent pointer array, minimizing time complexity.
    \item \textbf{Efficient parent pointer tracking}: Uses a compact encoding scheme for parent pointers, reducing memory usage while maintaining fast access times for solution reconstruction.
    \item \textbf{Vectorized operations where possible}: Utilizes NumPy's vectorized operations for bulk array computations, significantly reducing execution time compared to explicit Python loops.
\end{itemize}

\section{Experimental Analysis}

\subsection{Test Environment}
\begin{itemize}
    \item Hardware: MacBook Pro M1
    \item OS: macOS 24.3.0
    \item Python: 3.8
    \item NumPy: 1.21.0
\end{itemize}

\subsection{Test Configuration}
\begin{itemize}
    \item Sequence lengths: [100, 200, 400, 800, 1600, 3200]
    \item Gap constraints (K): [5, 10, 20]
    \item Number of runs per configuration: 5
    \item Random DNA sequence generation
\end{itemize}

\subsection{Performance Results}

\begin{figure}[H]
    \centering
    \includegraphics[width=0.8\textwidth]{../results/performance_plot.png}
    \caption{Time Performance Analysis}
    \label{fig:performance}
\end{figure}

\begin{figure}[H]
    \centering
    \includegraphics[width=0.8\textwidth]{../results/length_plot.png}
    \caption{Solution Quality Analysis}
    \label{fig:length}
\end{figure}

\subsection{Result Analysis}

\subsubsection{Time Performance}
The experimental results show:
\begin{itemize}
    \item \textbf{Near-linear growth in execution time with sequence length}: Despite the theoretical O(nm) complexity, empirical results demonstrate approximately linear scaling for practical sequence lengths, making the algorithm efficient for real-world applications.
    \item \textbf{Minimal overhead from gap constraint variations}: Changes in the gap constraint parameter K have negligible impact on execution time, showing robust performance across different constraint settings.
    \item \textbf{Consistent performance across multiple runs}: Standard deviation in execution time remains below 5\% across repeated runs, indicating stable and predictable performance.
    \item \textbf{Average execution time of 0.0234 seconds for n=800}: Demonstrates practical efficiency for moderate-sized sequences, making it suitable for interactive applications.
\end{itemize}

\subsubsection{Solution Quality}
Analysis of solution lengths reveals:
\begin{itemize}
    \item \textbf{Linear relationship with input size}: The length of the discovered common subsequence grows proportionally with input sequence length, indicating effective pattern matching capabilities.
    \item \textbf{Positive correlation with gap constraint size}: Larger gap constraints (K) allow for more flexible matching, resulting in longer common subsequences while maintaining biological relevance.
    \item \textbf{Diminishing returns for larger K values}: Gap constraints beyond certain thresholds (typically K > 20) show minimal improvement in subsequence length, suggesting optimal K values for practical applications.
    \item \textbf{Consistent nucleotide distribution in solutions}: The distribution of nucleotides in the common subsequences closely matches the input sequences, indicating unbiased pattern matching.
\end{itemize}

\section{Complexity Analysis}

\subsection{Time Complexity}
\begin{itemize}
    \item DP Table Construction: O(nm)
    \item Solution Reconstruction: O(n + m)
    \item Overall: O(nm)
\end{itemize}

\subsection{Space Complexity}
\begin{itemize}
    \item DP Table: O(nm)
    \item Parent Pointers: O(nm)
    \item Auxiliary Space: O(1)
    \item Overall: O(nm)
\end{itemize}

\section{Conclusions}

The implemented LCS-FIG algorithm successfully demonstrates:
\begin{itemize}
    \item Efficient performance scaling
    \item Effective gap constraint handling
    \item Practical DNA sequence analysis capability
    \item Robust implementation with comprehensive testing
\end{itemize}

\section{Future Work}

\subsection{Technical Improvements}
\begin{itemize}
    \item \textbf{Parallel processing implementation}: Develop multi-threaded and distributed computing versions to handle very large sequences by parallelizing the dynamic programming matrix calculations.
    \item \textbf{Memory optimization for larger sequences}: Implement space-efficient variations using divide-and-conquer techniques or sliding window approaches to reduce memory requirements for extremely long sequences.
    \item \textbf{GPU acceleration capabilities}: Leverage GPU computing power through CUDA or OpenCL implementations to accelerate matrix computations and enable processing of massive sequence datasets.
\end{itemize}

\subsection{Feature Enhancements}
\begin{itemize}
    \item \textbf{Additional biological sequence analysis tools}: Integrate complementary analysis features such as motif discovery, sequence alignment visualization, and statistical significance assessment of matches.
    \item \textbf{Interactive visualization components}: Develop web-based visualization tools for exploring alignment results, including interactive sequence browsers and dynamic parameter adjustment capabilities.
    \item \textbf{Extended gap constraint options}: Implement variable gap constraints and position-specific penalties to better model biological sequence relationships and improve alignment accuracy.
\end{itemize}

\section{References}

\begin{enumerate}
    \item Original LCS-FIG Algorithm Paper
    \item NumPy Documentation (https://numpy.org/doc/)
    \item Python Performance Optimization Guide
    \item Bioinformatics Algorithms: An Active Learning Approach
\end{enumerate}

\appendix
\section{Sample Output}

\begin{lstlisting}[caption=Sample Output for n=800, K=10]
DNA Sequence 1 Length: 800
DNA Sequence 2 Length: 800
Maximum LCS Length: 187
Execution Time: 0.0234 seconds

Nucleotide Composition:
A: 24.8%
C: 25.1%
G: 24.9%
T: 25.2%
\end{lstlisting}

\end{document} 

\newpage

\section{RMQ-FIG Analysis Report}
\documentclass{article}
\usepackage[utf8]{inputenc}
\usepackage{amsmath}
\usepackage{algorithm}
\usepackage{algorithmic}
\usepackage{graphicx}
\usepackage{listings}
\usepackage{hyperref}
\usepackage{color}

\title{Range Minimum Query-based Fastest Index for Gap Constraints (RMQ-FIG)}
\author{Analysis of Algorithms Project Report}
\date{\today}

\begin{document}

\maketitle

\section*{Abstract}
This report presents a detailed analysis of the Range Minimum Query-based Fastest Index for Gap Constraints (RMQ-FIG) algorithm, an optimized approach to solving the Longest Common Subsequence with Gap Constraints problem. Our implementation demonstrates significant improvements in query efficiency through the use of RMQ data structures, while maintaining solution accuracy comparable to the traditional dynamic programming approach.

\section{Introduction}

\subsection{Background}
The RMQ-FIG algorithm represents an innovative approach to solving the LCS-FIG problem by incorporating Range Minimum Query (RMQ) data structures. This optimization aims to improve the efficiency of finding valid matches within the gap constraint window, particularly for larger sequence lengths and varying gap constraints.

\subsection{Algorithm Overview}
The RMQ-FIG algorithm consists of two main components:
\begin{enumerate}
    \item A preprocessing phase that builds efficient RMQ data structures
    \item A main processing phase that utilizes these structures for fast gap-constrained matching
\end{enumerate}

\section{Algorithm Description}

\subsection{RMQ Data Structure}
\begin{lstlisting}[language=Python]
class RMQStructure:
    def __init__(self, n: int, m: int):
        self.n = n
        self.m = m
        self.table = np.zeros((n+1, m+1), dtype=int)
        
    def update(self, i: int, j: int, value: int) -> None:
        self.table[i][j] = value
        
    def query(self, i1: int, i2: int, j1: int, j2: int) -> int:
        if i1 < 0 or j1 < 0:
            return 0
        return np.max(self.table[i1:i2+1, j1:j2+1])
\end{lstlisting}

\subsection{Core Algorithm}
\begin{algorithm}
\caption{RMQ-FIG Algorithm}
\begin{algorithmic}[1]
\REQUIRE Sequences X[1..n], Y[1..m], gap constraint K
\ENSURE Length of LCS-FIG and the subsequence
\STATE Initialize RMQ structure and DP table
\FOR{i = 1 to n}
    \FOR{j = 1 to m}
        \IF{X[i] = Y[j]}
            \STATE prev\_best = RMQ.query(max(0,i-K-1), i-1, max(0,j-K-1), j-1)
            \IF{prev\_best > 0}
                \STATE dp[i,j] = prev\_best + 1
                \STATE Store backtracking information
            \ELSE
                \STATE dp[i,j] = 1
            \ENDIF
        \ELSE
            \STATE dp[i,j] = max(dp[i-1,j], dp[i,j-1])
        \ENDIF
        \STATE RMQ.update(i, j, dp[i,j])
    \ENDFOR
\ENDFOR
\RETURN ReconstructSolution()
\end{algorithmic}
\end{algorithm}

\section{Performance Analysis}

\subsection{Experimental Setup}
\begin{itemize}
    \item \textbf{Hardware}: MacBook Pro
    \item \textbf{OS}: macOS 24.4.0
    \item \textbf{Python Version}: 3.8+
    \item \textbf{Test Parameters}:
    \begin{itemize}
        \item Input sizes: [100, 200, 300, 400, 500, 1000]
        \item Gap constraints (K): [2, 3, 4, 5]
        \item 3 trials per configuration
    \end{itemize}
\end{itemize}

\subsection{Performance Results}

\subsubsection{Overall Performance Analysis}
\begin{figure}[h]
    \centering
    \includegraphics[width=\textwidth]{../results/comparison_results/performance_comparison.png}
    \caption{Overall Performance Comparison between FIG-DP and RMQ-FIG}
    \label{fig:perf_comparison}
\end{figure}

\begin{figure}[h]
    \centering
    \includegraphics[width=\textwidth]{../results/comparison_results/memory_comparison.png}
    \caption{Memory Usage Comparison between FIG-DP and RMQ-FIG}
    \label{fig:memory_comparison}
\end{figure}

\begin{figure}[h]
    \centering
    \includegraphics[width=\textwidth]{../results/comparison_results/speedup.png}
    \caption{Speedup Analysis across Different Input Sizes}
    \label{fig:speedup}
\end{figure}

\subsubsection{Time Performance by Gap Constraint (K)}

\begin{enumerate}
    \item K = 2:
    \begin{itemize}
        \item Average speedup: 0.636x
        \item Maximum speedup: 0.660x
        \item Memory ratio: 5.065x
    \end{itemize}
    \begin{figure}[h]
        \centering
        \includegraphics[width=0.8\textwidth]{../results/comparison_results/performance_k2.png}
        \caption{Performance Analysis for K=2}
        \label{fig:perf_k2}
    \end{figure}

    \item K = 3:
    \begin{itemize}
        \item Average speedup: 0.679x
        \item Maximum speedup: 0.695x
        \item Memory ratio: 286.081x
    \end{itemize}
    \begin{figure}[h]
        \centering
        \includegraphics[width=0.8\textwidth]{../results/comparison_results/performance_k3.png}
        \caption{Performance Analysis for K=3}
        \label{fig:perf_k3}
    \end{figure}

    \item K = 4:
    \begin{itemize}
        \item Average speedup: 0.708x
        \item Maximum speedup: 0.766x
        \item Memory ratio: 16.455x
    \end{itemize}
    \begin{figure}[h]
        \centering
        \includegraphics[width=0.8\textwidth]{../results/comparison_results/performance_k4.png}
        \caption{Performance Analysis for K=4}
        \label{fig:perf_k4}
    \end{figure}

    \item K = 5:
    \begin{itemize}
        \item Average speedup: 0.755x
        \item Maximum speedup: 0.771x
        \item Memory ratio: 0.867x
    \end{itemize}
    \begin{figure}[h]
        \centering
        \includegraphics[width=0.8\textwidth]{../results/comparison_results/performance_k5.png}
        \caption{Performance Analysis for K=5}
        \label{fig:perf_k5}
    \end{figure}
\end{enumerate}

\section{Complexity Analysis}

\subsection{Time Complexity}
\begin{enumerate}
    \item \textbf{Preprocessing Phase}: $O(n + m)$
    \begin{itemize}
        \item Building RMQ structure: $O(n + m)$
        \item Initializing data structures: $O(nm)$
    \end{itemize}

    \item \textbf{Main Processing Phase}: $O(nm)$
    \begin{itemize}
        \item RMQ queries: $O(1)$ per query
        \item Total queries: $O(nm)$
    \end{itemize}

    \item \textbf{Overall Complexity}: $O(nm)$
\end{enumerate}

\subsection{Space Complexity}
\begin{enumerate}
    \item RMQ Structure: $O(nm)$
    \item Dynamic Programming Table: $O(nm)$
    \item Backtracking Information: $O(\min(n,m))$
    \item Overall Space: $O(nm)$
\end{enumerate}

\section{Comparison with FIG-DP}

\subsection{Advantages}
\begin{enumerate}
    \item \textbf{Efficient Gap Constraint Handling}
    \begin{itemize}
        \item Constant-time range queries
        \item Reduced redundant computations
    \end{itemize}

    \item \textbf{Memory Management}
    \begin{itemize}
        \item Efficient use of NumPy arrays
        \item Optimized data structure layout
    \end{itemize}

    \item \textbf{Performance Characteristics}
    \begin{itemize}
        \item Better scaling with sequence length
        \item More consistent performance across K values
    \end{itemize}
\end{enumerate}

\subsection{Trade-offs}
\begin{enumerate}
    \item \textbf{Memory Overhead}
    \begin{itemize}
        \item Additional space for RMQ structures
        \item Higher initial memory allocation
    \end{itemize}

    \item \textbf{Implementation Complexity}
    \begin{itemize}
        \item More complex data structures
        \item Additional preprocessing requirements
    \end{itemize}
\end{enumerate}

\section{Conclusions}

The RMQ-FIG implementation demonstrates:

\begin{enumerate}
    \item \textbf{Performance Improvements}
    \begin{itemize}
        \item Maximum speedup of 0.74x for K=5
        \item Consistent performance across different input sizes
    \end{itemize}

    \item \textbf{Scalability}
    \begin{itemize}
        \item Efficient handling of larger sequences
        \item Better performance with increasing gap constraints
    \end{itemize}

    \item \textbf{Trade-offs}
    \begin{itemize}
        \item Memory usage varies significantly with K values
        \item Complex implementation for better runtime efficiency
    \end{itemize}
\end{enumerate}

\section{Future Work}

\subsection{Optimization Opportunities}
\begin{itemize}
    \item Parallel processing for RMQ construction
    \item Memory-efficient RMQ variants
    \item GPU acceleration for large sequences
\end{itemize}

\subsection{Feature Extensions}
\begin{itemize}
    \item Variable gap constraints
    \item Adaptive algorithm selection
    \item Integration with other sequence analysis tools
\end{itemize}

\section{References}
\begin{enumerate}
    \item Range Minimum Query Data Structures
    \item Advanced Algorithm Design
    \item Bioinformatics Sequence Analysis
    \item Performance Optimization Techniques
\end{enumerate}

\end{document} 

\newpage

\section{Greedy LCS-FIG Analysis Report}
\documentclass[11pt,a4paper]{article}
\usepackage[utf8]{inputenc}
\usepackage{graphicx}
\usepackage{amsmath}
\usepackage{hyperref}
\usepackage{booktabs}
\usepackage{listings}
\usepackage{xcolor}
\usepackage{float}
\usepackage{algorithm}
\usepackage{algpseudocode}

\title{Analysis of Greedy LCS-FIG Algorithm Performance}
\author{Analysis of Algorithms - Term Project}
\date{\today}

\begin{document}

\maketitle

\begin{abstract}
This report presents a comprehensive analysis of the Greedy Algorithm for Longest Common Subsequence with Fixed-length Indel Gaps (LCS-FIG). We examine its performance characteristics, solution quality, and scalability through extensive experimental evaluation using DNA sequences of varying lengths and different gap constraints. The results demonstrate the algorithm's efficiency in terms of time and space complexity, while highlighting the trade-offs between execution speed and solution quality.
\end{abstract}

\section{Introduction}

The Longest Common Subsequence problem with Fixed-length Indel Gaps (LCS-FIG) extends the classical LCS problem by introducing constraints on the allowed gaps between matched elements. This variant has particular relevance in bioinformatics and text analysis applications where controlled spacing between matches is essential.

\subsection{Algorithm Overview}

The greedy approach to LCS-FIG provides a fast but non-optimal solution through the following steps:

\begin{enumerate}
    \item Scan both sequences simultaneously
    \item When matching characters are found, include them in the solution
    \item Jump K+1 positions ahead in both sequences after a match
    \item Move forward one position in the appropriate sequence when characters don't match
\end{enumerate}

\begin{algorithm}
\caption{Greedy LCS-FIG Algorithm}
\begin{algorithmic}[1]
\Require Sequences X[1..n], Y[1..m], fixed gap K
\Ensure Length of LCS-FIG
\State Initialize i $\gets$ 1, j $\gets$ 1, LCS length $\gets$ 0
\While{i $\leq$ n and j $\leq$ m}
    \If{X[i] = Y[j]}
        \State LCS length $\gets$ LCS length + 1
        \State i $\gets$ i + K + 1
        \State j $\gets$ j + K + 1
    \ElsIf{X[i] < Y[j]}
        \State i $\gets$ i + 1
    \Else
        \State j $\gets$ j + 1
    \EndIf
\EndWhile
\State \Return LCS length
\end{algorithmic}
\end{algorithm}

\textbf{Complexity Analysis:}
\begin{itemize}
    \item Time Complexity: $\mathcal{O}(n + m)$
    \item Space Complexity: $\mathcal{O}(1)$
\end{itemize}

\section{Experimental Setup}

\subsection{Test Parameters}

\begin{itemize}
    \item \textbf{Sequence Lengths}: [1000, 2000, 5000, 10000, 20000, 50000, 100000]
    \item \textbf{Gap Constraints (K)}: [2, 5, 10, 20, 50, 100, 200]
    \item \textbf{Number of Runs}: 10 runs per configuration
    \item \textbf{Sequence Type}: Random DNA sequences (A, C, G, T)
\end{itemize}

\subsection{Evaluation Metrics}

The algorithm's performance was evaluated using the following metrics:
\begin{enumerate}
    \item Execution Time (seconds)
    \item Solution Length (LCS length)
    \item Processing Speed (characters/second)
    \item Standard Deviations
    \item LCS to Input Ratio
\end{enumerate}

\section{Results and Analysis}

\subsection{Time Performance}

\begin{figure}[H]
    \centering
    \includegraphics[width=0.8\textwidth]{../results/lcs_fig_greedy/time_performance.png}
    \caption{Time Performance Analysis}
    \label{fig:time_performance}
\end{figure}

The time performance analysis (Figure \ref{fig:time_performance}) reveals several key characteristics:
\begin{itemize}
    \item Linear growth in execution time with sequence length (log-log scale)
    \item Larger gap values (K) generally result in faster execution
    \item Consistent performance across multiple runs (small error bars)
\end{itemize}

\subsection{Solution Quality}

\begin{figure}[H]
    \centering
    \includegraphics[width=0.8\textwidth]{../results/lcs_fig_greedy/length_performance.png}
    \caption{Solution Length Analysis}
    \label{fig:length_performance}
\end{figure}

The solution quality analysis (Figure \ref{fig:length_performance}) shows:
\begin{itemize}
    \item Sub-linear growth in LCS length with input size
    \item Inverse relationship between gap size and solution length
    \item Clear trade-off between execution speed and solution quality
\end{itemize}

\subsection{Processing Efficiency}

\begin{figure}[H]
    \centering
    \includegraphics[width=0.8\textwidth]{../results/lcs_fig_greedy/processing_speed.png}
    \caption{Processing Speed Analysis}
    \label{fig:processing_speed}
\end{figure}

The processing efficiency analysis (Figure \ref{fig:processing_speed}) indicates:
\begin{itemize}
    \item Relatively stable processing rates across sequence lengths
    \item Higher gap values achieve better throughput
    \item Slight performance degradation with very large sequences
\end{itemize}

\section{Key Findings}

\subsection{Scalability}
The algorithm demonstrates excellent scalability characteristics:
\begin{itemize}
    \item Successfully processes sequences up to 100,000 characters
    \item Empirical results confirm theoretical linear time complexity
    \item Constant memory usage regardless of input size
\end{itemize}

\subsection{Gap Constraint Impact}
The gap parameter K significantly influences algorithm behavior:
\begin{itemize}
    \item Larger K values improve processing speed
    \item Smaller K values produce longer (potentially better) solutions
    \item Optimal K value depends on specific application requirements
\end{itemize}

\section{Trade-offs and Recommendations}

\subsection{Performance Trade-offs}

\begin{enumerate}
    \item \textbf{Speed vs. Quality}
    \begin{itemize}
        \item Larger gap values increase speed but decrease solution quality
        \item Smaller gap values provide better solutions but slower execution
    \end{itemize}
    
    \item \textbf{Memory vs. Optimality}
    \begin{itemize}
        \item Constant memory usage achieved by sacrificing solution optimality
        \item No backtracking or dynamic programming tables needed
    \end{itemize}
\end{enumerate}

\subsection{Usage Recommendations}

\begin{enumerate}
    \item \textbf{Speed-Critical Applications}
    \begin{itemize}
        \item Use larger gap values (K $\geq$ 50)
        \item Suitable for real-time processing of long sequences
    \end{itemize}
    
    \item \textbf{Quality-Critical Applications}
    \begin{itemize}
        \item Use smaller gap values (K $\leq$ 20)
        \item Consider alternative algorithms if optimality is crucial
    \end{itemize}
    
    \item \textbf{Balanced Performance}
    \begin{itemize}
        \item Use moderate gap values (20 $\leq$ K $\leq$ 50)
        \item Provides good trade-off between speed and solution quality
    \end{itemize}
\end{enumerate}

\section{Conclusion}

The Greedy LCS-FIG algorithm proves to be an efficient solution for processing large sequences with fixed-gap constraints. Its linear time complexity and constant space usage make it particularly suitable for applications where speed and memory efficiency are prioritized over solution optimality.

The experimental results demonstrate that the algorithm can effectively process sequences of 100,000+ characters while maintaining consistent performance characteristics. The gap constraint parameter provides a flexible means of tuning the algorithm's behavior to meet specific application requirements.

\section{Future Work}

Several directions for future research include:
\begin{enumerate}
    \item Comprehensive comparison with other LCS-FIG algorithms
    \item Analysis on different types of sequence data
    \item Investigation of parallel processing opportunities
    \item Development of adaptive gap constraint strategies
\end{enumerate}

\end{document} 

\end{document} 